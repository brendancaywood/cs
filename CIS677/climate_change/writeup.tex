\documentclass[11pt]{article}

\usepackage{listings}
\usepackage{graphicx}
\usepackage{tabularx}

\begin{document}
\title{Spark-Accelerated Climate Change Analysis}
\author{Jarred Parr}
\date{December 2018}
\maketitle

\section{Introduction}
Climate change is one of the most talked about topics in modern society. From the mainstream media sources, to the President of
the United States, you can't avoid the conversation. This project aims to do some analysis of some of the features of Earth that
would be more directly affected by climate change. The predicted results would primarily be increased wind temperature, and more
severe weather, a product of which being higher wind speeds. This project does analysis on the wind speed variations over the
years and the wind temperature variations. This is in the pursuit of providing a reasonable model to show the affect that climate
change has had in even a short time frame as 3 decades. A linear regressor was used to analyze the data and help determine if,
based on the data, our current trend will likely continue in whatever direction it was found to go, or if it will being to
slowly normalize over time.

\section{Program Architecture}
The software itself is very barebones. The PySpark library takes much of the guesswork out of how to implement things and gives
an easy to use and refreshing interface in which to interact with the underlying structures in place. As a result, the code itself
was extrememly easy to implement and work with. Results were able to be obtained in the use of just two functions which,
realistically, could have been made into just one. There was nothing particularly exemplary discovered in the way of code design
other than that the implementation itself was rather clean and straightforward.

\section{Code}
The code is as follows:

\lstset{frame=tb, language=python}
\begin{lstlisting}
print('Placeholder text')
\end{lstlisting}

\section{Results}
\begin{table}[htb]
\begin{tabular}{lllll}
 1  &  2  &  \\
 1  &  2  &  \\
 1  &  2  &  \\
 1  &  2  &
\end{tabular}
\end{table}

\end{document}
